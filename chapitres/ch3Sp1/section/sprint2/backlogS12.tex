Dans cette section, nous présentons le backlog du sprint 1.2, tel qu'illustré dans le tableau \ref{tab:backlogS22}.
Ce backlog détaille les besoins sélectionnées pour ce sprint, accompagnées de leurs tâches associées, priorités, risques et estimations en jours.
\begin{landscape}
    \renewcommand{\arraystretch}{1.3}
    \begin{spacing}{0.94}
        \begin{longtable}{|p{0.6cm}|p{2.6cm}|p{4.9cm}|p{0.97cm}|p{8.6cm}|p{0.35cm}|p{0.35cm}|p{1.6cm}|}
            \caption{Backlog du sprint 1.2} \label{tab:backlogS22} \\\hline
            \rowcolor{gray!20}
            \textbf{\small ID US} & 
            \multicolumn{1}{c|}{\textbf{\small User Story}} & 
            \multicolumn{1}{c|}{\textbf{\small Description}} & 
            \textbf{\small ID tâche}& 
            \multicolumn{1}{c|}{\textbf{\small Tâches}} & 
            \multicolumn{1}{c|}{\textbf{\small Priorité}} & 
            \multicolumn{1}{c|}{\textbf{\small Risques}} & 
            \textbf{\fontsize{9}{11}\selectfont Estimation (Jours)}\\\hline
            % ----------- SCANS DE SÉCURITÉ ------------------
			\rowcolor{blue!20}
            \multicolumn{8}{|c|}{\textbf{EPIC 4: Gestion des scans de tests de sécurité d’un site web}} \\\hline
            
            4.1 & Configurer les paramètres de scan selon les besoins. & En tant que testeur, je veux personnaliser les paramètres de scan pour adapter les analyses aux besoins spécifiques. 
            & 4.1.A \newline\vspace{0.5cm} 4.1.B &
            - Développer une interface pour configurer les paramètres de scan. \newline
            - Permettre à l’utilisateur de choisir la profondeur des scans. & Moyenne & Moyenne & 1/2\\ \hline

           4.2 & Sélectionner les outils de sécurité à utiliser pour l’analyse.
            & En tant que testeur, je veux choisir les outils à utiliser afin d’adapter l’analyse aux besoins de la cible, enregistrer mes préférences et les réutiliser lors des scans suivants.
             & 4.2.A \newline\vspace{1.9cm} 4.2.B&
            - Développer une interface avec des cases à cocher permettant de sélectionner les outils souhaités avant le lancement d’un scan avec une sélection multiple et des boutons « Tout sélectionner » / « Tout désélectionner ».\newline
            - Enregistrer les outils préférés de chaque utilisateur dans la base de données et les recharger automatiquement pour les scans suivants.
            & Élevée  & Élevée  & 1\\ \hline
            
            4.3 & Lancer un scan de test de sécurité. & En tant que testeur, je veux initier un scan de test de sécurité sur une cible pour identifier ses vulnérabilités. 
            & 
            4.3.A \newline\vspace{1cm} 
            4.3.B\newline\vspace{0.9cm} 
            4.3.C\newline\vspace{0.5cm} 
            4.3.D\newline\vspace{0.5cm} 
            4.3.E\newline\vspace{0.5cm} 
            4.3.F\newline\vspace{0.5cm} 
            4.3.G&
            - Développer une interface de lancement de scan avec champ URL de la cible à tester, les outils, paramètres de scan et bouton "Lancer".\newline
            - Corriger l'automatisation des outils existants \textbf{ZAP} et \textbf{Wapiti} en vérifiant leur configuration et optimisant les paramètres de détection.\newline
            - Intégrer et automatiser l’exécution d’outils spécialisés tels que SQLMap, Nuclei, Nmap...\newline
            - Utiliser le multithreading pour exécuter les outils en parallèle.\newline
            - Unifier les formats de sortie pour générer un rapport commun.\newline
            - Créer une base de données des vulnérabilités détectées.\newline
            - Générer un rapport JSON unifié via un modèle d’agrégation et de comparaison.
            & Élevée & Moyenne & 1 \\ \hline
            
            4.4 & Lancer un scan de sécurité avec authentification dynamique. 
                & En tant que testeur, je souhaite initier un scan authentifié afin d’identifier les vulnérabilités présentes dans les zones protégées.  & 4.4.A \newline\vspace{0.5cm} 4.4.B  & 
                - Intégrer l’authentification dynamique pour chaque outil. \newline
                - Tester les mécanismes d’authentification pour chaque outil (cookies, jetons, mots de passe). & Élevée & Moyenne &3 \\ \hline
            
            4.5 & Planifier des scans  de test sécurité automatiques. 
                & En tant que testeur, je veux définir une planification automatique des scans pour assurer une surveillance régulière. & 4.5.A \newline\vspace{0.5cm} 4.5.B  &
                - Créer une interface de planification pour automatiser les scans. \newline
                - Tester le bon déroulement des scans planifiés. & Élevée & Moyenne & 1 \\ \hline
                
            4.6 & Suivre la progression du scan en temps réel via WebSocket. 
                & En tant que testeur, je veux visualiser en temps réel l’évolution des scans pour suivre leur progression.
                & 4.6.A \newline 4.6.B &
                - Intégrer WebSocket pour suivre la progression. \newline
                - Mettre à jour l'interface utilisateur avec des informations en temps réel.  & Élevée & Moyenne & 1 \\ \hline            
            4.7 & Visualiser les résultats des scans. 
                    & En tant que testeur, je peux consulter les résultats afin d'analyser la sécurité de l'application. 
                    & 4.7.A \newline\vspace{0.5cm} 4.7.B
                    & 
                    - Implémenter une interface pour visualiser les résultats des scans. \newline
                    - Fournir des options de filtrage pour faciliter l'analyse des résultats. 
                    & Élevée & Basse & 2 \\
                \hline
            4.8 & Accéder à l’historique des scans précédents. 
                    & En tant que testeur, je dois accéder aux rapports des anciens scans pour suivre l'évolution des vulnérabilités et conserver une trace des analyses précédentes.
                    & 4.8.A 
                    \newline 4.8.B
                    \newline\vspace{0.5cm} 4.8.C
                    \newline 4.8.D
                    \newline\vspace{0.4cm} 4.8.E
                    \newline\vspace{0.5cm} 4.8.F
                    \newline\vspace{0.cm} 4.8.G
                    & 
                     - Afficher les historiques de scans.\newline
                     - Ajouter une pagination pour naviguer entre les pages de résultats.\newline
                     - Ajouter des filtres (par type, outil, gravité...). \newline
                     - Implémenter une barre de recherche pour retrouver un scan précis.\newline
                     - Ajouter une option de suppression pour chaque rapport.\newline
                     - Permettre l'accès aux détails d’un scan : liste des vulnérabilités par outil et les logs associés à chaque scan.\newline
                     - Afficher un résumé global du scan accompagné de la liste complète des vulnérabilités détectées pour chaque scan.
                    & Moyenne & Basse & 1 \\
                \hline
                4.9 & Télécharger les rapports aux formats JSON, PDF et CSV. 
                    & En tant que testeur, je dois télécharger les rapports sous différents formats pour faciliter leur traitement et archivage. 
                    & 4.9.A \newline\vspace{0.5cm} 4.9.B
                    & 
                    - Développer des options d'exportation pour les rapports. \newline
                    - Ajouter des boutons pour télécharger les rapports en formats JSON, PDF et CSV.
                    & Moyenne & Moyenne & 1 \\
                \hline
                4.10 & Intégrer et visualiser les rapports via Jira. 
                    & En tant que testeur, je dois intégrer les résultats dans Jira pour créer des tickets et assurer un suivi structuré des incidents détectés. 
                    & 4.10.A \newline\vspace{0.5cm} 4.10.B
                    & 
                    - Créer une intégration avec Jira pour la création automatique de tickets. \newline
                    - Visualiser les résultats dans des dashboards Jira.
                    & Élevée & Élevée & 3/2 \\
                \hline
                4.11 & Accéder aux rapports via Slack. 
                    & En tant que testeur, je dois recevoir les rapports via Slack pour une communication rapide au sein de l'équipe. 
                    & 4.11.A \newline\vspace{0.5cm} 4.11.B
                    & 
                    - Mettre en place une intégration avec Slack pour envoyer les rapports. \newline
                    - Ajouter des notifications Slack pour chaque scan terminé.
                    & Moyenne & Basse & 3/2 \\
                \hline
                4.12 & Recevoir les rapports directement par e-mail. 
                    & En tant que testeur, je dois recevoir automatiquement les rapports par e-mail pour assurer leur disponibilité hors plateforme. 
                    & 4.12.A \newline\vspace{0.5cm}4.12.B
                    & 
                    - Configurer l'envoi automatique de rapports par e-mail. \newline
                    - Ajouter un modèle d'e-mail pour l'envoi des rapports.
                    & Moyenne & Basse & 3/2 \\
                \hline
                4.13 & Détecter automatiquement les pages de login, en excluant les pages d’inscription. 
                    & En tant que testeur, je souhaite que le système identifie automatiquement les pages d’authentification d’un site pour configurer correctement les scans authentifiés. 
                    & 4.13.A \newline\vspace{0.5cm} 4.13.B
                    & 
                    - Analyser le code HTML des pages pour détecter les formulaires de connexion.\newline
                    - Mettre en place des règles pour exclure les pages d’inscription et les pages non pertinentes.
                    & Élevée & Moyenne & 2 \\
                \hline
                4.14 & Annuler un scan de sécurité en cours. 
                    & En tant que testeur, je souhaite pouvoir annuler un scan de sécurité en cours d'exécution afin d’arrêter une analyse inutile ou incorrectement configurée.
                    & 4.14.A \newline\vspace{0.5cm} 4.14.B
                    & 
                    - Ajouter un bouton "Annuler" dans l'interface de suivi en temps réel du scan.\newline
                    - Implémenter la logique backend pour interrompre proprement l'exécution des outils lancés (thread/process/containers).
                    & Élevée & Moyenne & 2 \\
                \hline
                4.14 & Relancer un scan à partir de la configuration précédente.
                & En tant que testeur, je souhaite relancer facilement un scan de sécurité en utilisant les paramètres d’un ancien scan pour gagner du temps et assurer la reproductibilité des tests.
                & 4.14.A \newline\vspace{0.5cm} 4.14.B 
                &
                - Permettre la duplication d’un scan depuis l’historique avec récupération automatique de la configuration (outils, paramètres, cible, type d’authentification, etc.). \newline
                - Développer une interface "Relancer ce scan" accessible depuis les détails d’un scan précédent. \newline
                & Moyenne & Moyenne & 1 \\
            \hline

        % ----------- Notifications ------------------
           \rowcolor{blue!20}
           \multicolumn{8}{|c|}{\textbf{EPIC 9: Notifications en temps réel}} \\\hline
            9.1 & Être notifié des résultats pendant l’exécution des scans. 
                & En tant que testeur, je dois recevoir des notifications immédiates pour suivre l'état des scans et détecter rapidement les incidents. 
                & 9.1.A \newline\vspace{0.5cm} 9.1.B 
                &
                - Configurer le système de notifications en temps réel. \newline
                - Tester l'envoi de notifications pendant l'exécution des scans. 
                & Élevée & Basse & 1 \\ \hline
            9.2 & Envoyer des alertes pour les vulnérabilités critiques. 
                & En tant que testeur, je dois recevoir des alertes spécifiques pour les vulnérabilités critiques afin de pouvoir réagir rapidement. 
                & 9.2.A \newline\vspace{0.5cm} 9.2.B 
                &
                - Définir les critères de vulnérabilités critiques pour l'envoi d'alertes. \newline
                - Automatiser l'envoi des alertes en fonction de la gravité des vulnérabilités détectées. 
                & Élevée & Élevée & 1 \\ \hline
            \hline   
        \rowcolor{blue!20}
           \multicolumn{8}{|c|}{\textbf{EPIC 10: Paramétrage des canaux de diffusion des rapports de tests}} \\\hline
                10.1 & Définir les identifiants et jetons d’accès pour Slack. 
                & En tant que testeur, je veux configurer les identifiants d’accès Slack pour activer l’envoi automatique des rapports dans les canaux de l’équipe.
                & 10.1.A \newline\vspace{0.5cm}10.1.B
                & - Ajouter un formulaire de saisie des tokens Slack.\newline - Tester l’envoi automatisé d’un rapport via Slack.
                & Élevée & Moyenne & 1/2\\\hline
                
                10.2 & Saisir les adresses e-mail des destinataires. 
                & En tant qu’utilisateur, je veux définir les adresses email des destinataires pour permettre la diffusion automatique des rapports.
                & 10.2.A \newline\vspace{0.5cm}10.2.B
                & - Créer une interface de configuration des adresses e-mail.\newline - Tester l’envoi de rapports PDF par e-mail.
                & Moyenne & Faible & 1/2\\\hline
                
                10.3 & Configurer les paramètres d’intégration Jira. 
                & En tant qu’administrateur, je veux paramétrer l’URL, l’ID projet et les clés API Jira pour permettre la création automatisée de tickets avec les résultats de scan.
                & 10.3.A \newline\vspace{0.5cm}10.3.B
                & - Développer une interface de saisie des paramètres Jira.\newline - Tester l’intégration avec la création d’un ticket depuis un rapport.
                & Élevée & Moyenne & 1/2\\\hline
                
                10.4 & Sélectionner les types et formats de rapports à envoyer. 
                & En tant qu’utilisateur, je souhaite choisir quels types (sécurité, fonctionnel, SEO) et quels formats (HTML, PDF, JSON) de rapports seront transmis.
                & 10.4.A \newline\vspace{0.5cm}10.4.B
                & - Implémenter une interface pour sélectionner les types et formats de rapport.\newline - Tester l’envoi avec les différentes combinaisons sélectionnées.
                & Moyenne & Moyenne & 1/4\\\hline
                
                10.5 & Activer ou désactiver les canaux de notification. 
                & En tant qu’utilisateur, je veux activer ou désactiver les notifications Slack, Email ou Jira selon mes préférences.
                & 10.5.A
                & - Ajouter des boutons d’activation/désactivation pour chaque canal.\newline
                & Faible & Faible & 1/4\\\hline
                

            \rowcolor{gray!20}
			\multicolumn{7}{|c|}{TOTAL} &  24 (Jours)\\
            \hline 
        \end{longtable}
    \end{spacing}
    \vspace{-0.1cm}
\end{landscape}


