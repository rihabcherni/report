Afin de surmonter les nombreuses limites identifiées dans l’application existante, la solution proposée vise à développer une plateforme complète pour l’automatisation des tests de sécurité, des audits SEO, et des tests fonctionnels avec une analyse fiable et complète des résultats. Elle vise à renforcer la visibilité des applications tout en garantissant leur bon fonctionnement. Cette nouvelle solution repose sur une approche intégrée et évolutive avec les objectifs suivants:
\begin{itemize}[label=$\bullet$,left=0.09cm]
    \item \textbf{Intégration optimisée des outils de pénétration}: Utilisation de ZAP\cite{zap}, Wapiti\cite{wapiti}, Nuclei\cite{nuclei}, Nikto\cite{nikto}, SQLMap\cite{sqlmap}, XSStrike\cite{xsstrike} et autres, chacun spécialisé dans une famille de vulnérabilités, pour détecter les failles de sécurité et couvrir un périmètre d’analyse plus large et plus fiable.
    
    \item \textbf{Mécanisme de comparaison, de consolidation des résultats et de validation automatique des faux positifs}: regroupement des vulnérabilités similaires détectées par différents outils, avec une pondération selon leur niveau de risque, afin d’éliminer les doublons, de clarifier les rapports, d’améliorer leur précision, de filtrer les résultats pertinents, de réduire les erreurs de détection et d’optimiser le temps d’analyse.

    \item \textbf{Scan d’authentification automatisé}: Détection des formulaires de connexion après identification des champs login/password, cookies et tokens d’authentification pour permettre les scans dans les zones protégées en reproduisant des scénarios d’attaque.
    
    \item \textbf{Automatisation des tests fonctionnels}: en exécutant des scénarios de tests simulant le comportement utilisateur (remplissage de formulaires, clics, redirections), afin de vérifier que les fonctionnalités clés restent opérationnelles et conformes aux exigences des utilisateurs.
    
    \item \textbf{Automatisation des tests SEO}: pour analyser les techniques du référencement (temps de chargement, structure HTML, balises, liens cassés).
    
    \item \textbf{Système de reporting}: Génération de rapports synthétiques et clairs avec des filtres (par outil, par niveau de gravité, par catégorie), visualisation graphique interactive, export en PDF, HTML, JSON ou CSV, envoi par e-mail, Slack ou Jira.
    \item \textbf{Refonte de la base de données} : adoption d’une structure relationnelle optimisée avec des tables normalisées permettant un stockage détaillé des rapports, une gestion granulaire des vulnérabilités ainsi qu’une meilleure exploitation des données via des requêtes avancées.
    \item \textbf{Gestion intelligente des scans concurrents} : Mise en place d’un système de file d’attente et de verrouillage par utilisateur afin d’empêcher le lancement simultané ou successif non contrôlé de plusieurs scans. Une logique de planification permet d’exécuter les scans de manière ordonnée, tout en assurant un contrôle d’accès aux ressources partagées. Des vérifications en temps réel préviennent les doublons, les surcharges ou les blocages. Des statuts d’exécution offrent à l’utilisateur la possibilité de suivre l’avancement d’un scan ou de l’annuler si nécessaire.
    \item \textbf{Interface responsive et modernisée}: Refonte complète du frontend en Angular dernière version 18, accès contrôlé par authentification JWT\cite{jwt} et rôle utilisateur, et responsive design pour mobile et tablette.
    \item \textbf{Sécurisation complète des accès} : Mise en place d’un contrôle d’accès uniforme sur l’ensemble des routes, avec redirections automatiques, gestion des sessions et mécanisme de réinitialisation de mot de passe.
\end{itemize}
Cette solution permettra de corriger en profondeur les failles de l’application actuelle en rendant les processus de tests efficaces et compréhensibles. Elle contribuera aussi à fournir une plateforme unifiée adaptée aux besoins réels des développeurs, des testeurs et des responsables sécurité.
