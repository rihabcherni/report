Pour la planification, nous avons choisi de répartir le travail en deux livraisons (releases), chacune étant divisée en deux sprints, comme indiqué dans le tableau \ref{tab:Planification}.
{\renewcommand{\arraystretch}{1.3}
\begin{table}[H]
	\begin{center}
    \caption{Planification des Livraisons et des Sprints}
    \label{tab:Planification}
	\begin{tabular}{|p{8.2cm}|p{8.2cm}|}\hline
		\textbf{Livraison 1 (Total : 50 jours)} & \textbf{Livraison 2 (Total : 47 jours)} \\\hline
		\begin{minipage}[t]{\linewidth}
            \textbf{Sprint 1.1} (26 jours) :
            \begin{itemize}[label=$-$, left=0.2cm]
                \item Initialisation du projet
                \item Consultation de la page d’accueil
                \item Authentification
                \item Gestion des utilisateurs
            \end{itemize}
            \textbf{Sprint 1.2} (24 jours) :
            \begin{itemize}[label=$-$, left=0.2cm]
                \item Gestion des scans de tests de sécurité
                \item Paramétrage des canaux de diffusion des rapports de tests
                \item Notifications en temps réel
            \end{itemize}
            \vspace{0.1cm}
        \end{minipage}
        &
        \begin{minipage}[t]{\linewidth}
            \textbf{Sprint 2.1} (25 jours) :
            \begin{itemize}[label=$-$, left=0.2cm]
                \item Gestion des scans fonctionnels
                \item Gestion des scans SEO
            \end{itemize}
            \textbf{Sprint 2.2} (22 jours) :
            \begin{itemize}[label=$-$, left=0.2cm]
                \item Gestion des scans complètes (fonctionnels, sécurité et SEO)
                \item Visualisation du tableau de bord
                \item Gestion des rapports des analyses effectuées
                \item Déploiement de l’application
            \end{itemize}
        \end{minipage}
        \\\hline
	\end{tabular}
 	\end{center}
\end{table}
}
\vspace{-0.7cm}
Cette répartition a permis d’ajuster le projet au fil des sprints tout en intégrant les fonctionnalités essentielles dès les premières étapes du développement.