\begin{justify}
    Les besoins non fonctionnels définissent comment le système doit fonctionner incluant les contraintes d’implémentation, d’environnement, de performance, de dépendances, de maintenance, d’extensibilité, de sécurité, de fiabilité et d’ergonomie. Une analyse approfondie de ces besoins est essentielle pour garantir la qualité globale et l’efficacité d’un produit logiciel\cite{bnf}.

    Pour optimiser l’expérience utilisateur dans ce projet, il est important de respecter les exigences de qualité suivantes:
     \begin{itemize}[label=$\bullet$, left=0.15cm]
        \item \textbf{Sécurité:} Le système doit garantir que les données des utilisateurs ainsi que les résultats des scans soient stockés de manière sécurisée. Les rapports doivent être protégés et accessibles uniquement aux utilisateurs autorisés.
        \item \textbf{Scalabilité:} Le système doit être capable de gérer une charge importante de tests effectués simultanément, sans dégradation de la performance. Il doit permettre un traitement parallèle efficace, où chaque test s'exécute de manière indépendante, garantissant que plusieurs utilisateurs puissent fonctionner simultanément sans interférer les uns avec les autres.
        \item \textbf{Fiabilité:} Le système doit être conçu pour être tolérant aux pannes et permettre une reprise automatique des scans en cas de défaillance. De plus, il doit garantir l'envoi immédiat d'alertes en temps réel lorsqu'un problème est détecté, comme des vulnérabilités critiques durant un scan, afin de permettre une réaction rapide et appropriée.
        \item \textbf{Disponibilité:} Le système doit être disponible en permanence, garantissant un fonctionnement sans interruption, 24h/24 et 7j/7.
        \item \textbf{UX/UI intuitive:} L'interface utilisateur doit être claire et ergonomique, permettant une navigation fluide et intuitive pour tous les utilisateurs. L'objectif est d'assurer une interaction simple et rapide, sans courbe d'apprentissage complexe, avec des retours immédiats lors des actions et une navigation optimisée.
        \item \textbf{Compatibilité et adaptabilité:} L'interface doit être responsive, s’adaptant de manière optimale à toutes les tailles et résolutions d’écrans, tout en offrant une expérience cohérente sur tous les navigateurs et plateformes, quel que soit l'appareil utilisé.
    \end{itemize}
\end{justify}   