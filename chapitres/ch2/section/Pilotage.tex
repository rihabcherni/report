Cette section décrit le pilotage du projet selon la méthodologie Scrum, en présentant les outils utilisés, la composition de l’équipe et ses rôles, le backlog du produit et le découpage du projet.
\subsection{Présentation de l’équipe de travail }
La réussite du projet repose sur une bonne organisation de l’équipe et une répartition claire des rôles, assurées par la méthodologie Scrum. Dans le cadre de ce projet, les trois rôles clés de Scrum sont attribués aux membres de l’équipe de la manière suivante :
\begin{itemize}[label=$\bullet$]
    \item \textbf{Scrum Master:} Représenté par notre encadrant professionnel, Monsieur \textbf{Omar NACHMI}, qui a guidé l’équipe dans l’adoption de Scrum, assurant une organisation efficace et une amélioration continue des processus de travail.
    \item \textbf{Product Owner :}  Représenté également par Monsieur \textbf{Omar NACHMI}, qui a proposé et dirigé ce travail, en fournissant des explications approfondies sur les diverses exigences et fonctionnalités requises par le système.
    \item \textbf{Équipe de développement:} Composée de moi-même, \textbf{Rihab Cherni}, et de ma collègue \textbf{Maïssa Ben Ghalba}, nous avons travaillé ensemble pour réaliser le projet, en collaborant étroitement à chaque étape du développement.
\end{itemize}
\subsection{Outils SCRUM utilisés}
Pour le suivi de notre projet, nous avons adopté des outils collaboratifs : \textbf{OpenProject} pour la gestion des tâches et le suivi quotidien de notre progression, \textbf{Microsoft Teams} pour la communication entre les membres de l’équipe, l’organisation des réunions quotidiennes (daily meetings), des réunions RH et la coordination des sprints, et \textbf{GitLab} pour le travail collaboratif sur le code et la gestion de version.
 \subsection{Backlog du produit }
Le backlog produit représente une liste des fonctionnalités nécessaires au développement du produit. Géré par le Product Owner, il permet d'assurer l’alignement entre les besoins des utilisateurs et les objectifs du projet. Il constitue un outil de référence pour l’équipe de développement, en collaboration avec le Scrum Master et les parties prenantes \cite{backlog}.

Les epics structurent le backlog à un niveau stratégique en regroupant des fonctionnalités clés du produit. Plus génériques et moins détaillés que les user stories, ils doivent être découpés progressivement pour être développés. Ils facilitent la planification, la priorisation, la communication avec les parties prenantes et la gestion de la complexité tout au long du projet\cite{epic}.

Le tableau \ref{tab:backlog} présente le backlog produit de notre projet, en précisant pour chaque épique sa priorité, le niveau de risque, ainsi que son effort estimé en jours.
\renewcommand{\arraystretch}{1.3}
\begin{spacing}{1}
    \begin{longtable}{|p{0.35cm}|p{9.3cm}|p{1.45cm}|p{1.45cm}|p{2.4cm}|}
        \caption{Backlog produit} \label{tab:backlog} \\\hline
        \textbf{\small ID} & \textbf{\small Épics} & \textbf{\small Priorité} & \textbf{\small Risques} & \textbf{\small Estimation(j)} \\ \hline
        1  & \textbf{Initialisation du projet} & Élevée & Basse & 15 \\
        \hline
        2 & \textbf{Consultation de la page d’accueil} & Basse & Basse & 5 \\
        \hline 
        3  & \textbf{Authentification et gestion du profil} & Moyenne & Basse & 4 \\
        \hline
       4  & \textbf{Gestion des scans de tests de sécurité d’un site web} & Élevée & Élevée & 20 \\
        \hline  
        5  & \textbf{Gestion des tests fonctionnels d’un site web} & Élevée & Élevée & 15 \\
        \hline 
        6  & \textbf{Gestion des analyses SEO d’un site web} & Élevée & Moyenne & 10 \\
        \hline      
        7  & \textbf{Gestion des analyses complètes} & Élevée & Élevée & 6 \\
        \hline
        8  & \textbf{Visualisation des statistiques via le tableau de bord} & Élevée & Élevée & 8 \\
        \hline
        9  & \textbf{Notifications en temps réel} & Élevée & Basse & 2 \\
        \hline
        10 & \textbf{Paramétrage des canaux de diffusion des rapports de tests} & Moyenne & Moyenne & 2 \\ 
        \hline
        11 & \textbf{Gestion des utilisateurs} & Moyenne & Basse & 2 \\
        \hline
        12 & \textbf{Gestion des rapports des analyses effectuées } & Moyenne & Basse & 3 \\
        \hline
        13 & \textbf{Déploiement de l'application} & Moyenne & Moyenne & 5 \\
        \hline
        \multicolumn{4}{|c|}{\textbf{TOTAL}} & \textbf{97(jours)} \\
        \hline 
    \end{longtable}
\end{spacing}
\vspace{-0.1cm}
\subsection{Planification des sprints}
Pour la planification, nous avons choisi de répartir le travail en deux livraisons (releases), chacune étant divisée en deux sprints, comme indiqué dans le tableau \ref{tab:Planification}.
{\renewcommand{\arraystretch}{1.3}
\begin{table}[H]
	\begin{center}
    \caption{Planification des Livraisons et des Sprints}
    \label{tab:Planification}
	\begin{tabular}{|p{8.2cm}|p{8.2cm}|}\hline
		\textbf{Livraison 1 (Total : 50 jours)} & \textbf{Livraison 2 (Total : 47 jours)} \\\hline
		\begin{minipage}[t]{\linewidth}
            \textbf{Sprint 1.1} (26 jours) :
            \begin{itemize}[label=$-$, left=0.2cm]
                \item Initialisation du projet
                \item Consultation de la page d’accueil
                \item Authentification
                \item Gestion des utilisateurs
            \end{itemize}
            \textbf{Sprint 1.2} (24 jours) :
            \begin{itemize}[label=$-$, left=0.2cm]
                \item Gestion des scans de tests de sécurité
                \item Paramétrage des canaux de diffusion des rapports de tests
                \item Notifications en temps réel
            \end{itemize}
            \vspace{0.1cm}
        \end{minipage}
        &
        \begin{minipage}[t]{\linewidth}
            \textbf{Sprint 2.1} (25 jours) :
            \begin{itemize}[label=$-$, left=0.2cm]
                \item Gestion des scans fonctionnels
                \item Gestion des scans SEO
            \end{itemize}
            \textbf{Sprint 2.2} (22 jours) :
            \begin{itemize}[label=$-$, left=0.2cm]
                \item Gestion des scans complètes (fonctionnels, sécurité et SEO)
                \item Visualisation du tableau de bord
                \item Gestion des rapports des analyses effectuées
                \item Déploiement de l’application
            \end{itemize}
        \end{minipage}
        \\\hline
	\end{tabular}
 	\end{center}
\end{table}
}
\vspace{-0.7cm}
Cette répartition a permis d’ajuster le projet au fil des sprints tout en intégrant les fonctionnalités essentielles dès les premières étapes du développement.