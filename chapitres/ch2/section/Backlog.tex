Le backlog produit représente une liste des fonctionnalités nécessaires au développement du produit. Géré par le Product Owner, il permet d'assurer l’alignement entre les besoins des utilisateurs et les objectifs du projet. Il constitue un outil de référence pour l’équipe de développement, en collaboration avec le Scrum Master et les parties prenantes \cite{backlog}.

Les epics structurent le backlog à un niveau stratégique en regroupant des fonctionnalités clés du produit. Plus génériques et moins détaillés que les user stories, ils doivent être découpés progressivement pour être développés. Ils facilitent la planification, la priorisation, la communication avec les parties prenantes et la gestion de la complexité tout au long du projet\cite{epic}.

Le tableau \ref{tab:backlog} présente le backlog produit de notre projet, en précisant pour chaque épique sa priorité, le niveau de risque, ainsi que son effort estimé en jours.
\renewcommand{\arraystretch}{1.3}
\begin{spacing}{1}
    \begin{longtable}{|p{0.35cm}|p{9.3cm}|p{1.45cm}|p{1.45cm}|p{2.4cm}|}
        \caption{Backlog produit} \label{tab:backlog} \\\hline
        \textbf{\small ID} & \textbf{\small Épics} & \textbf{\small Priorité} & \textbf{\small Risques} & \textbf{\small Estimation(j)} \\ \hline
        1  & \textbf{Initialisation du projet} & Élevée & Basse & 15 \\
        \hline
        2 & \textbf{Consultation de la page d’accueil} & Basse & Basse & 5 \\
        \hline 
        3  & \textbf{Authentification et gestion du profil} & Moyenne & Basse & 4 \\
        \hline
       4  & \textbf{Gestion des scans de tests de sécurité d’un site web} & Élevée & Élevée & 20 \\
        \hline  
        5  & \textbf{Gestion des tests fonctionnels d’un site web} & Élevée & Élevée & 15 \\
        \hline 
        6  & \textbf{Gestion des analyses SEO d’un site web} & Élevée & Moyenne & 10 \\
        \hline      
        7  & \textbf{Gestion des analyses complètes} & Élevée & Élevée & 6 \\
        \hline
        8  & \textbf{Visualisation des statistiques via le tableau de bord} & Élevée & Élevée & 8 \\
        \hline
        9  & \textbf{Notifications en temps réel} & Élevée & Basse & 2 \\
        \hline
        10 & \textbf{Paramétrage des canaux de diffusion des rapports de tests} & Moyenne & Moyenne & 2 \\ 
        \hline
        11 & \textbf{Gestion des utilisateurs} & Moyenne & Basse & 2 \\
        \hline
        12 & \textbf{Gestion des rapports des analyses effectuées } & Moyenne & Basse & 3 \\
        \hline
        13 & \textbf{Déploiement de l'application} & Moyenne & Moyenne & 5 \\
        \hline
        \multicolumn{4}{|c|}{\textbf{TOTAL}} & \textbf{97(jours)} \\
        \hline 
    \end{longtable}
\end{spacing}
\vspace{-0.1cm}