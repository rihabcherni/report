\chapter*{\begin{center} \huge \vspace{-3.5cm} \textbf{Conclusion générale} \end{center}}
\addcontentsline{toc}{chapter}{Conclusion générale}
\vspace{-2.5cm}
\begin{justify}
Ce projet de fin d’études nous a permis de répondre à un besoin croissant dans le domaine de la cybersécurité et de la qualité web : l’automatisation des tests de sécurité, des tests fonctionnels et des audits SEO. En adoptant une démarche agile (Scrum) et en utilisant des technologies modernes comme Angular, FastAPI et PostgreSQL, nous avons conçu une application web performante, évolutive et maintenable.

Ce travail nous a permis de mieux appréhender les enjeux liés à la sécurité et à la qualité des applications web, notamment à travers la détection automatisée de vulnérabilités, la validation des faux positifs et l’amélioration de la visibilité sur les moteurs de recherche. Les résultats obtenus ont confirmé la pertinence de l’approche, avec des tests fiables, reproductibles, et des rapports clairs.

Au-delà des aspects techniques, ce projet a renforcé nos compétences en développement full-stack, en sécurité informatique, en gestion agile et en automatisation des tests. Il constitue une étape clé dans notre parcours d’ingénieur, ouvrant la voie à des évolutions futures.

Parmi les pistes d’amélioration envisagées :
\begin{itemize}[label=$-$]
    \item \textbf{Intégration de l’intelligence artificielle :} pour optimiser l’analyse des résultats, détecter les vulnérabilités, anticiper les failles et générer automatiquement des scénarios de tests.

    \item \textbf{Extension aux tests mobiles :} prise en charge des applications mobiles (Android/iOS) afin d’évaluer leur sécurité, leur performance et leur conformité aux bonnes pratiques.
    
    \item \textbf{Extension des audits SEO :} avec des critères avancés comme l’expérience utilisateur ou la performance mobile.
    
    \item \textbf{Support des environnements cloud :} exécution des tests sur des infrastructures distribuées, intégration avec les pipelines CI/CD.
        
    \item \textbf{Support multilingue :} pour rendre l’application accessible à un public international.
\end{itemize}

En somme, ce projet constitue une base solide pour le développement d’une solution complète, adaptable et intégrée dans des processus de développement sécurisés à grande échelle.

\end{justify}
