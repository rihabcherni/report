\chapter*{\begin{center} \huge \vspace{-3.5cm} \textbf{Introduction générale} \end{center}}
\addcontentsline{toc}{chapter}{Introduction générale}
\vspace{-2.5cm}
    \begin{justify}
        La sécurité des applications web constitue un enjeu majeur face à l’augmentation des cyberattaques, dont les vulnérabilités peuvent entraîner la perte de données sensibles. Cependant, les méthodes traditionnelles de détection de ces failles restent souvent manuelles, lentes, limitées, coûteuses et sujettes aux erreurs humaines. Dans ce contexte, les tests de pénétration jouent un rôle crucial en permettant d’identifier les failles avant qu’elles ne soient exploitées par des attaquants.
        
        En parallèle, les tests fonctionnels vérifient que les fonctionnalités d’une application respectent les spécifications et fonctionnent correctement dans divers scénarios. Ils permettent de s’assurer que les interactions, les flux de travail, les formulaires, les boutons, et les diverses actions proposées à l’utilisateur se comportent comme prévu. Bien que souvent réalisés manuellement, leur automatisation permet un gain de temps, une meilleure couverture des tests et une réduction des erreurs humaines. Cela facilite la détection rapide des dysfonctionnements, améliore la qualité globale de l’application et assure une expérience utilisateur fluide. Ces tests sont essentiels pour garantir la robustesse du produit, prévenir les bugs et renforcer la confiance des utilisateurs.   
        
        De plus, l’optimisation du référencement naturel (audit SEO) est essentielle pour évaluer et améliorer la visibilité d’un site ou d’une application web sur les moteurs de recherche. Elle permet d’identifier les points forts, les faiblesses et les axes d’amélioration du positionnement. L’automatisation de ces tests permet de détecter rapidement ces problèmes et d’optimiser la qualité technique globale.
    
        L’objectif de ce projet est de concevoir et développer une application web pour automatiser :
        \begin{itemize}[left=-0.1cm, label=$\bullet$]
            \item \textbf{Les tests de pénétration :} Intégrer divers outils de sécurité afin d'améliorer la détection des vulnérabilités, qu’elles soient similaires ou complémentaires, dans le cadre des tests d'intrusion. Cela inclut la gestion de la progression des scans, la comparaison et la corrélation des vulnérabilités, la fusion des résultats en un rapport final unifié, la surveillance en temps réel des vulnérabilités détectées, l’implémentation d’une authentification dynamique pour simuler des scénarios d’attaque sur des applications protégées par des pages de connexion, ainsi que la validation automatique des faux positifs.
            \item \textbf{Les tests fonctionnels :} Automatiser les tests pour vérifier que les fonctionnalités principales des applications web fonctionnent correctement et tester les interfaces utilisateur pour s'assurer de leur conformité avec les attentes.
            \item \textbf{Les tests SEO :} Identifier automatiquement les erreurs techniques pouvant affecter le référencement naturel, telles que les liens cassés, les redirections défaillantes, l’absence ou la mauvaise configuration des balises essentielles, ainsi qu’un temps de chargement excessif, qui peuvent nuire à l’indexation et à la performance globale. Fournir ensuite des recommandations pour améliorer la visibilité de l’application sur les moteurs de recherche.
        \end{itemize}
    
        Le rapport présente les étapes du projet, structuré comme suit :
        \begin{itemize}[left=-0.1cm, label=$\bullet$]
            \item \textbf{Chapitre 1 : "Cadre général du projet" :} Contexte général du projet, présentation de la société d’accueil, analyse des solutions existantes sur le marché et la méthodologie de travail.
            \item \textbf{Chapitre 2 : "Sprint 0 - Étude préliminaire" :} Analyse des besoins fonctionnels et non fonctionnels, découpage du projet et élaboration du backlog produit. Présentation exhaustive des outils et technologies adoptés lors de l’implémentation, ainsi que de l’architecture de l’application.
            \item \textbf{Chapitre 2 : "Analyse et planification du projet" :} Analyse des besoins fonctionnels et non fonctionnels, découpage du projet et élaboration du backlog produit, ainsi que présentation détaillée des outils et technologies utilisés lors de l’implémentation et de l’architecture de l’application.
            \item \textbf{Chapitre 3 : "Sprint 1 -  Automatisation des tests de sécurité et amélioration des fonctionnalités de base.}
            \item \textbf{Chapitre 4 : "Sprint 2 - Automatisation des tests fonctionnels et SEO, génération de rapports et déploiement.".}
        \end{itemize}
        
        Enfin, une conclusion générale résumera les résultats réalisés et proposera des perspectives d’amélioration, tout en énumérant les compétences que nous avons acquises durant ce stage.
    \end{justify}
    
    
